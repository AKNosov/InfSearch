\section{Итоги работы}

В рамках курса «Информационный поиск» разработана система информационного поиска по новостным статьям, включающая сбор корпуса, хранение в MongoDB и поисковый движок на C++ для булевого поиска:

\begin{enumerate}
    \item Реализован поисковый робот на Python для автоматического сбора статей из источников \texttt{lenta.ru} и \texttt{rbc.ru}.
    
    \item Создана инфраструктура Docker/Docker Compose с персистентным хранением корпуса в MongoDB (volume), что обеспечивает сохранность данных при перезапусках.
    
    \item Реализовано устойчивое хранение состояния обхода: очередь URL и прогресс робота хранятся в MongoDB, поэтому робот может быть остановлен и продолжает работу после повторного запуска.
    
    \item Реализована периодическая переобкачка документов: страницы переобходятся по расписанию и обновляются только при изменении (по хэшу контента).
    
    \item Реализован токенизатор на C++ с поддержкой кириллицы и латиницы (UTF-8) по правилам:
    разделители --- пробелы и пунктуация; длина токена 2..50; приведение к нижнему регистру; числа сохраняются; URL и email пропускаются; дефисы и апострофы обрабатываются корректно.
    
    \item Реализован стеммер на C++ на основе алгоритма Портера:
    \begin{itemize}
        \item русский язык --- Snowball/Porter Russian stemmer;
        \item английский язык --- Porter stemmer.
    \end{itemize}
    
    \item Реализована собственная хеш-таблица для хранения инвертированного индекса (open addressing + linear probing) без использования \texttt{std::unordered\_map}.
    
    \item Построен булев инвертированный индекс (терм $\rightarrow$ posting list docId), загружаемый и строящийся непосредственно по корпусу из MongoDB.
    
    \item Реализован булев поиск с поддержкой \texttt{AND}, \texttt{OR}, \texttt{NOT} и скобок, а также неявного \texttt{AND} между соседними термами.
    
    \item Выполнен анализ закона Ципфа для корпуса: построен log-log график rank--frequency и оценён показатель степени $\alpha$.
    
    \item Разработаны unit-тесты для токенизатора и стеммера (отдельные тестовые файлы), а также интеграционный сценарий ``MongoDB $\rightarrow$ C++ движок''.
\end{enumerate}

\subsection{Оценка качества работы}

\textbf{Достоинства реализации:}
\begin{itemize}
    \item Корпус хранится в MongoDB с персистентностью и может обновляться без потери данных.
    \item Робот возобновляет работу после остановки за счёт очереди в MongoDB.
    \item Автоматическая дедупликация документов по нормализованному URL (уникальный индекс).
    \item Вежливое ограничение частоты запросов (rate limiting) и повторные попытки при ошибках.
    \item Поисковый движок реализован с ограничениями по STL: ключевые структуры (хеш-таблица, индекс) реализованы самостоятельно.
    \item Эффективные булевы операции над отсортированными posting lists.
    \item Поддержка UTF-8 и нормализация токенов (кириллица/латиница, дефисы, апострофы, числа).
\end{itemize}

\textbf{Недостатки и ограничения:}
\begin{itemize}
    \item Зависимость от структуры сайтов-источников: при изменении верстки/URL-шаблонов требуется корректировка фильтров и селекторов извлечения текста.
    \item Индекс строится в памяти при запуске движка; для больших объёмов требуется больше RAM или сохранение индекса на диск.
    \item Отсутствует ранжирование результатов (TF-IDF, BM25): поиск является булевым и возвращает множество документов без сортировки по релевантности.
    \item Нет фразового поиска и позиционного индекса.
\end{itemize}

\subsection{Направления развития}

Возможные улучшения системы:

\begin{enumerate}
    \item Добавить сохранение инвертированного индекса на диск и инкрементальное обновление при изменении корпуса.
    \item Реализовать ранжирование результатов (TF-IDF / BM25) и хранение частот термов.
    \item Реализовать позиционный индекс для фразового поиска.
    \item Добавить поддержку расширенных запросов (wildcard/префиксный поиск).
    \item Добавить кэширование популярных запросов и статистику использования.
    \item Улучшить извлечение текста статей (домен-специфичные извлекатели для \texttt{lenta.ru} и \texttt{rbc.ru}).
    \item Расширить набор источников (добавить третий независимый источник при необходимости).
    \item Добавить веб-интерфейс (Flask) для удобного ввода запросов и просмотра результатов.
\end{enumerate}

\subsection{Приложения}

\begin{enumerate}
    \item Репозиторий проекта: \url{https://github.com/AKNosov/InfSearch}
    \item Гугл диск с корпусом документов: \url{https://drive.google.com/drive/folders/12kmarbw8LzpsjQn4KfKwM2wCCRxjZQby}
\end{enumerate}

\subsection{Список литературы}

\begin{enumerate}
    \item Manning C. D., Raghavan P., Schütze H. \textit{Introduction to Information Retrieval}. Cambridge University Press, 2008.
    \item Croft W. B., Metzler D., Strohman T. \textit{Search Engines: Information Retrieval in Practice}. Addison-Wesley, 2010.
    \item \textit{MongoDB Manual}. \url{https://www.mongodb.com/docs/} (дата обращения: \today).
\end{enumerate}

