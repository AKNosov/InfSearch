
\section{Задание}

В рамках курса «Информационный поиск» необходимо разработать полнофункциональную систему информационного поиска, включающую следующие компоненты:

\subsection{Сбор корпуса документов}

\begin{enumerate}
    \item Подготовить корпус документов из нескольких источников
    \item Обеспечить постоянное хранение и возможность обновления корпуса
    \item Реализовать автоматическую систему сбора данных 
\end{enumerate}

Требования к корпусу:
\begin{itemize}
    \item Размер: минимум 30,000 документов (для оценки «удовлетворительно»)
    \item Источники: минимум 2 независимых источника
    \item Формат хранения: MongoDB
\end{itemize}

\subsection{Индексация и поиск}

Реализовать поисковый движок на C++ с ограничениями на использование STL (только \texttt{vector} и \texttt{string}):

\begin{enumerate}
    \item \textbf{Токенизация} --- разбиение текста на лексемы с поддержкой кириллицы и латиницы
    \item \textbf{Стемминг} --- приведение слов к основе для русского и английского языков
    \item \textbf{Хеш-таблица} --- собственная реализация для хранения инвертированного индекса
    \item \textbf{Инвертированный индекс} --- структура для быстрого поиска термов в документах
    \item \textbf{Булев поиск} --- поддержка операций AND, OR, NOT и скобок
    \item \textbf{Закон Ципфа} --- анализ распределения частот термов
\end{enumerate}



\pagebreak