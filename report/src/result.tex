
\section{Результаты}

\subsection{Статистика корпуса}

Корпус статей собирался роботом на Python и сохранялся в MongoDB (коллекция \texttt{pages}). Для получения статистики использовался скрипт \texttt{analysis/corpus\_stats.py}, который вычисляет основные метрики по полям \texttt{text}, \texttt{url}, \texttt{source}, \texttt{fetched\_at}.

\begin{table}[h]
\centering
\begin{tabular}{|l|r|}
\hline
\textbf{Параметр} & \textbf{Значение} \\
\hline
Всего документов & \texttt{30 147} \\
Размер БД MongoDB & \texttt{1370.76 МБ } \\
Средняя длина текста & \texttt{2107 символов} символов \\
Средняя длина заголовка & \texttt{61 символов  } символов \\
Временной диапазон & \texttt{2025} \\
\hline
\end{tabular}
\caption{Статистика собранного корпуса}
\end{table}

\begin{table}[h]
\centering
\begin{tabular}{|l|r|r|}
\hline
\textbf{Источник} & \textbf{Документов} & \textbf{Доля, \%} \\
\hline
lenta.ru & \texttt{18 092} & \texttt{40.0} \\
rbc.ru   & \texttt{12 055 } & \texttt{60.0} \\
\hline
\textbf{Итого} & \textbf{\texttt{30 147}} & 100.0 \\
\hline
\end{tabular}
\caption{Распределение документов по источникам}
\end{table}

\subsection{Характеристики индекса}

\begin{table}[h]
\centering
\begin{tabular}{|l|r|}
\hline
\textbf{Параметр} & \textbf{Значение} \\
\hline
Количество документов & \texttt{30 147} \\
Количество уникальных термов & \texttt{89 835} \\
Формат индекса & in-memory (в памяти) \\
Время индексации & \texttt{299.88} сек \\
Скорость индексации & \texttt{101} док/сек \\
\hline
\end{tabular}
\caption{Характеристики индекса}
\end{table}


Поисковый движок на C++ загружает тексты из MongoDB и строит булевый инвертированный индекс в памяти (терм $\rightarrow$ posting list). Индекс хранится в собственной хеш-таблице (open addressing), posting lists сортируются для эффективных булевых операций.

\pagebreak

\subsection{Анализ закона Ципфа}

Для корпуса построен график распределения частот термов по рангу в логарифмической шкале (log-log). Дополнительно выполнена аппроксимация степенной функцией:

\[
f(r) = \frac{C}{r^\alpha}.
\]

Для построения использовался скрипт \texttt{analysis/zipf.py}, читающий тексты из MongoDB.

\begin{figure}[h]
\centering
\includegraphics[width=0.8\textwidth]{zipf.png}
\caption{Закон Ципфа: log(rank) vs log(frequency) для корпуса}
\end{figure}

\textbf{Причины расхождения с идеальным законом Ципфа:}
\begin{itemize}
    \item высокая доля служебных слов в ``голове'' распределения (предлоги/союзы), если не применяются стоп-слова;
    \item морфология русского языка: без лемматизации/при частичном стемминге ``хвост'' становится тяжелее;
    \item смешение тематик и источников (lenta/rbc) создаёт смесь распределений;
    \item наличие имён собственных, чисел, аббревиатур и редких токенов увеличивает количество hapax legomena.
\end{itemize}

\subsection{Примеры работы системы}

Поиск выполняется поверх MongoDB: сначала строится индекс, затем вводятся булевы запросы.


./engine mongodb://mongo:27017 crawler pages

Indexed: 30147 docs

Index build time: 6.38 sec

Speed: 4724 docs/sec

Boolean search ready.

> (нефть OR газ) AND NOT европа

hits: 318

  https://lenta.ru/news/2025/12/28/...
  
  https://lenta.ru/news/2025/12/27/...
  
  https://rbc.ru/economics/28/12/2025/...
  
  https://rbc.ru/politics/27/12/2025/...


\subsection{Производительность поиска}

Оценка времени выполнения запросов проводилась в рамках C++ движка после построения индекса в памяти. Быстродействие определяется:
\begin{itemize}
    \item временем операций пересечения/объединения posting lists;
    \item длиной списков для наиболее частотных термов;
    \item количеством операторов и скобок в запросе.
\end{itemize}

\begin{table}[h]
\centering
\begin{tabular}{|l|r|r|}
\hline
\textbf{Запрос} & \textbf{Результатов} & \textbf{Время, мкс} \\
\hline
нефть & 1847 & 82 \\
нефть AND газ & 412 & 104 \\
(нефть OR газ) AND NOT европа & 318 & 156 \\
\hline
\end{tabular}
\caption{Производительность булевого поиска}
\end{table}
\pagebreak

