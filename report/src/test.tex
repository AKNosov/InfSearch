
\section{Тестирование системы}

Для проверки корректности работы реализован набор unit- и интеграционных тестов, покрывающих ключевые компоненты: робота (сбор корпуса), движок (токенизация, стемминг, индекс, булев поиск) и работу поверх MongoDB.

\subsection{Тестирование робота (Python)}

\subsubsection{Корректность сохранения документа в MongoDB}
\textbf{Цель:} убедиться, что робот сохраняет документ в коллекцию \texttt{pages} с корректными полями.

\textbf{Шаги:}
\begin{enumerate}
    \item Запустить робота на ограниченном наборе URL (seed только на главную и одну статью).
    \item Проверить наличие документа в \texttt{pages}.
    \item Проверить наличие и типы полей: \texttt{url}, \texttt{text}, \texttt{source}, \texttt{fetched\_at}.
    \item Проверить, что \texttt{text} не пустой и имеет длину > 200 символов.
\end{enumerate}

\textbf{Результат:} документ присутствует в \texttt{pages}, поля заполнены, текст извлечён корректно.

\subsubsection{Возобновление работы после остановки}
\textbf{Цель:} проверить требование ``робота можно остановить и продолжить''.

\textbf{Шаги:}
\begin{enumerate}
    \item Запустить робота и дождаться заполнения \texttt{queue}.
    \item Остановить контейнер робота.
    \item Запустить контейнер робота снова.
    \item Проверить, что робот продолжает обработку из очереди, а не начинает обход с нуля.
\end{enumerate}

\textbf{Результат:} URL продолжают обрабатываться из \texttt{queue}, повторной генерации полной очереди не происходит.

\subsection{Unit-тесты поискового движка (C++)}

Unit-тесты реализованы для модулей \texttt{Tokenizer} и \texttt{Stemmer} в отдельных файлах:
\begin{itemize}
    \item \texttt{tests/tokenizer\_tests.cpp}
    \item \texttt{tests/stemmer\_tests.cpp}
\end{itemize}
\paragraph{}
Общий Unit-тест, кратко для всех модулей
\texttt{tests/stemmer\_tests.cpp} 

\subsubsection{Тесты токенизатора}

Покрываемые сценарии:
\begin{itemize}
    \item разделители: пробелы и пунктуация;
    \item приведение к нижнему регистру (кириллица и латиница), нормализация \texttt{ё->е};
    \item минимальная/максимальная длина токена (2..50);
    \item сохранение числовых токенов;
    \item пропуск URL и email (не индексируются);
    \item обработка дефисов и апострофов, включая Unicode-символы (\texttt{—}, \texttt{–}, \texttt{’}).
\end{itemize}

\subsubsection{Тесты стеммера}

Покрываемые сценарии:
\begin{itemize}
    \item английский Porter stemmer: классический набор примеров (caresses $\rightarrow$ caress, ponies $\rightarrow$ poni и т.п.);
    \item русский Porter/Snowball stemmer: группы словоформ, которые должны приводиться к одной основе;
    \item корректная работа на токенах с дефисом/апострофом (стемминг частей);
    \item безопасная обработка чисел и смешанных токенов (стеммер не должен ``падать'').
\end{itemize}


\subsection{Результаты тестирования}

Все unit-тесты для токенизации и стемминга, а также интеграционный сценарий ``движок поверх MongoDB'' выполнены успешно.

Пример вывода unit-тестов:
./tokenizer\_tests

[OK]   basic\_separators\_and\_lower

[OK]   numbers\_preserved

[OK]   min\_max\_len

[OK]   skip\_url\_and\_email

[OK]   hyphen\_kept\_inside\_word\_and\_parts\_present

[OK]   unicode\_dash\_is\_hyphen

[OK]   apostrophe\_handling\_ascii\_and\_unicode

[OK]   joiners\_at\_edges\_are\_delimiters

[OK]   yo\_to\_e\_and\_cyrillic\_upper\_to\_lower

ALL TOKENIZER TESTS PASSED
\paragraph{}
./stemmer\_tests

[OK]   english\_porter\_classic\_set

[OK]   russian\_same\_stem\_groups

[OK]   hyphen\_apostrophe\_parts\_are\_stemmed

[OK]   numbers\_and\_mixed\_tokens\_unchanged\_or\_safe

ALL STEMMER TESTS PASSED
\paragraph{}

./general\_tests

[OK]   tokenizer\_basic

[OK]   tokenizer\_min\_max\_len

[OK]   tokenizer\_skip\_url\_email

[OK]   tokenizer\_hyphen\_apostrophe

[OK]   stemmer\_english\_porter

[OK]   stemmer\_russian\_porter

[OK]   hashtable\_insert\_find

[OK]   hashtable\_rehash
[OK]   boolean\_index\_postings

[OK]   boolean\_search\_and\_or\_not\_parentheses

[OK]   boolean\_search\_implicit\_and

ALL TESTS PASSED


\pagebreak