\section{Описание решения}

\subsection{Архитектура системы}

Разработанная система информационного поиска состоит из следующих компонентов:

\begin{enumerate}
    \item \textbf{Поисковый робот (Python)} --- сбор и обновление корпуса документов из веб-источников.
    \item \textbf{База данных (MongoDB)} --- хранение корпуса статей, метаданных и служебной очереди обхода.
    \item \textbf{Поисковый движок (C++)} --- построение булевого инвертированного индекса по корпусу из MongoDB и выполнение булевых запросов.
    \item \textbf{Аналитический модуль (Python)} --- построение графика закона Ципфа по распределению частот терминов (опционально).
\end{enumerate}

\subsection{Выбор источников данных}

В рамках реализации используется новостной корпус, собираемый с двух независимых источников:

\begin{enumerate}
    \item \textbf{lenta.ru} --- новостной портал с большим объёмом регулярно обновляемых статей.
    \item \textbf{rbc.ru} --- крупное деловое СМИ с широкой тематикой и большим архивом публикаций.
\end{enumerate}

Выбор источников обусловлен следующими критериями:
\begin{itemize}
    \item Большой объём статей и высокая частота обновления.
    \item Наличие устойчивых URL-паттернов для статей (поддержка фильтрации служебных страниц).
    \item Достаточная тематическая широта для демонстрации работы булевого поиска.
\end{itemize}

\subsection{Формат хранения корпуса}

Корпус хранится в MongoDB в коллекции \texttt{pages}. Каждый документ содержит:
\begin{itemize}
    \item \texttt{url} --- нормализованный URL статьи (уникальный ключ);
    \item \texttt{html} --- исходный HTML (опционально, может сохраняться только при изменении);
    \item \texttt{text} --- чистый текст (заголовок + содержимое статьи);
    \item \texttt{source} --- источник (\texttt{lenta.ru} / \texttt{rbc.ru});
    \item \texttt{fetched\_at} --- время обкачки (Unix timestamp);
    \item \texttt{html\_hash} (или \texttt{content\_hash}) --- хэш для определения изменений и условной переобкачки.
\end{itemize}

Очередь обхода хранится в коллекции \texttt{queue} и содержит служебные поля \texttt{state}, \texttt{tries}, \texttt{next\_fetch\_at} и др., что позволяет:
\begin{itemize}
    \item останавливать робота в любой момент;
    \item продолжать работу с места остановки;
    \item выполнять периодическую переобкачку страниц согласно расписанию.
\end{itemize}

\subsection{Технологический стек}

\begin{itemize}
    \item \textbf{Python} --- реализация робота сбора корпуса, работа с HTTP, запись в MongoDB.
    \item \textbf{MongoDB} --- хранение корпуса и очереди обхода с персистентностью (Docker volume).
    \item \textbf{C++} --- реализация поискового движка (индексация и булев поиск). Ограничение по STL: используются только \texttt{vector} и \texttt{string}; хеш-таблица реализована самостоятельно.
    \item \textbf{Docker / Docker Compose} --- единый запуск компонентов и обеспечение воспроизводимости.
\end{itemize}

\subsection{Архитектура поискового робота}

Робот реализован как единый модуль на Python и использует MongoDB как устойчивое хранилище состояния (очереди). Основные функции робота:
\begin{itemize}
    \item нормализация URL;
    \item фильтрация служебных страниц и выделение URL статей;
    \item извлечение чистого текста статьи (заголовок + основной текст);
    \item добавление ссылок в очередь обхода;
    \item переобкачка документов по расписанию и обновление только при изменении (по хэшу).
\end{itemize}

\subsection{Арихтектура поискового движка}

\subsubsection{Инвертированный индекс (булев)}

Инвертированный индекс сопоставляет каждому терму список документов, в которых он встречается.

 В данной работе используется \textbf{булевый индекс}: терм добавляется в список документа один раз (без хранения частоты и позиций).

\subsubsection{Хеш-таблица}

Для хранения отображения \texttt{терм -> posting list} используется собственная реализация хеш-таблицы (open addressing + linear probing). Это обеспечивает амортизированную сложность доступа к posting list порядка $O(1)$.

\subsubsection{Токенизация}

Токенизация выполняется C++ модулем, который разбивает текст на лексемы (токены) по правилам:
\begin{itemize}
    \item разделители: пробелы и знаки пунктуации;
    \item минимальная длина токена: 2 символа;
    \item максимальная длина токена: 50 символов;
    \item приведение токенов к нижнему регистру (кириллица и латиница);
    \item числа сохраняются;
    \item URL и email пропускаются и не индексируются;

\end{itemize}

\subsubsection{Стемминг}

Стемминг приводит токены к основе:
\begin{itemize}
    \item русский язык: Snowball Russian stemmer;
    \item английский язык: алгоритм Портера (Porter stemmer).
\end{itemize}

\subsubsection{Булев поиск}

Булев поиск поддерживает операции \textbf{AND}, \textbf{OR}, \textbf{NOT} и круглые скобки. Запрос парсится в обратную польскую нотацию (алгоритм сортировочной станции / shunting-yard), после чего вычисляется над posting lists.

\paragraph{AND}
Пересечение двух отсортированных списков.



\paragraph{OR }
Объединение двух отсортированных списков аналогичным слиянием.

\paragraph{NOT }
Операция \texttt{NOT X} реализуется как дополнение множества документов \texttt{X} до универсального множества \texttt{AllDocs}.

\subsection{Закон Ципфа}

Для корпуса строится распределение частот термов по рангу и сравнивается с законом Ципфа:

\[
f(r) = \frac{C}{r^\alpha}, \quad \alpha \approx 1
\]

В логарифмическом масштабе зависимость линейна:

\[
\log f = \log C - \alpha \log r
\]

Для анализа используется Python-скрипт, который считывает тексты из MongoDB, токенизирует их, оценивает частоты и строит log-log график с наложением аппроксимации.

\pagebreak