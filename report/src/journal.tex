
\section{Журнал выполнения}

В данном разделе приведены основные проблемы, возникшие в процессе разработки системы (робот + MongoDB + C++ движок), и способы их решения.

\subsection{Проблема: Выделение страниц-статей и фильтрация служебных URL}

\textbf{Описание:} При обходе \texttt{lenta.ru} и \texttt{rbc.ru} большая часть ссылок ведёт на служебные страницы (рубрики, теги, поиск, авторизацию, медиа, рекламные вставки). При сохранении всех страниц корпус быстро засорялся нерелевантными документами.

\textbf{Решение:}
\begin{itemize}
    \item Введена нормализация URL (удаление фрагмента, tracking-параметров, унификация домена).
    \item Реализованы регулярные выражения для URL статей и чёрные списки путей (search/tag/auth/video/gallery и т.д.).
    \item В БД сохраняются только документы, удовлетворяющие критерию ``страница-статья'', однако ссылки на разделы допускаются в очереди для поиска новых статей.
\end{itemize}

\subsection{Проблема: Потеря прогресса обхода при остановке робота}

\textbf{Описание:} При остановке процесса робота обход начинался заново, что приводило к повторным запросам, замедлению и избыточной нагрузке на источники.

\textbf{Решение:}
\begin{itemize}
    \item Состояние робота перенесено в MongoDB: отдельная коллекция \texttt{queue} хранит URL, статус (\texttt{new/processing}), число попыток и время следующей обкачки \texttt{next\_fetch\_at}.
    \item Выбор следующей задачи выполнен атомарно через \texttt{find\_one\_and\_update}, поэтому после рестарта робот продолжает с места остановки.
\end{itemize}


\subsection{Проблема: Медленная скорость обхода и ``залипание'' на рубриках}

\textbf{Описание:} При ускорении робота потоками оказалось, что главная страница и рубрики вытесняют статьи: они часто переобнаруживаются и постоянно попадают в начало очереди. Это приводило к медленному росту числа документов в \texttt{pages}.

\textbf{Решение:}
\begin{itemize}
    \item Введено расписание переобкачки: разные интервалы для страниц-статей и для не-статейных страниц (рубрики/главная переобходятся чаще, статьи реже).
    \item Введён приоритет для очереди: статьи получают более высокий приоритет и выбираются в обработку раньше рубрик.
    \item Реализован rate limit по доменам (вежливое ограничение частоты запросов).
\end{itemize}

\subsection{Проблема: Подключение C++ движка к MongoDB в Docker-сети}

\textbf{Описание:} При первом запуске поискового движка из контейнера подключение к MongoDB не устанавливалось. Причина заключалась в использовании адреса \texttt{localhost} внутри контейнера, что указывает на сам контейнер, а не на сервис MongoDB в сети Docker Compose.

\textbf{Решение:}
\begin{itemize}
    \item В конфигурации запуска движка заменён URI подключения с \texttt{mongodb://localhost:27017} на \texttt{mongodb://mongo:27017}, где \texttt{mongo} --- имя сервиса MongoDB в \texttt{docker-compose.yml}.
    \item Для диагностики добавлены проверки доступности БД (вывод количества загруженных документов и времени чтения корпуса).
\end{itemize}

\subsection{Проблема: Индексация была слишком медленной из-за повторной обработки термов в документе}

\textbf{Описание:} На ранней версии индексатора термы добавлялись в posting list при каждом вхождении слова в документ. Это приводило к росту списков, необходимости последующей тяжёлой дедупликации и падению скорости индексации на больших текстах.

\textbf{Решение:}
\begin{itemize}
    \item Индекс переведён в \textbf{булевый} формат: внутри одного документа терм добавляется в posting list только один раз.
    \item Для дедупликации термов в пределах документа использована стратегия ``собрать все термы документа $\rightarrow$ отсортировать $\rightarrow$ удалить дубликаты $\rightarrow$ добавить docId в индекс''.
    \item После построения индекса выполняется финальная сортировка posting lists для корректных булевых операций.
\end{itemize}

\subsection{Проблема: Некорректная обработка булевых запросов со скобками и неявным AND}

\textbf{Описание:} На ранней версии поискового модуля запросы вида \texttt{нефть газ} или \texttt{(нефть OR газ) европа} интерпретировались неправильно: отсутствовала поддержка неявного оператора AND и корректный приоритет операций при наличии скобок. Это приводило к неверному числу результатов.

\textbf{Решение:}
\begin{itemize}
    \item Реализован полноценный парсер булевых выражений: лексер $\rightarrow$ преобразование в обратную польскую нотацию (алгоритм сортировочной станции) $\rightarrow$ вычисление выражения над posting lists.
    \item Добавлено правило неявного AND между соседними термами/скобками.
    \item Для проверки корректности добавлены unit-тесты на запросы с \texttt{AND/OR/NOT} и скобками.
\end{itemize}

\pagebreak